\documentclass[10pt,a4paper]{report}
\usepackage[pdftex]{graphicx}
\usepackage{subfigure}            % Multiple figures per float
\usepackage{placeins}             % Fix float loactions with \FloatBarrier
\usepackage{color}                % Color definitions for boxes
\usepackage{multirow}             % Multiple rows per cell in a table
\usepackage{verbatim}             % Long verbatim sections
\usepackage{titlesec}             % Easy modification the chapter format
\usepackage{fancyhdr}             % Easy modification of header and footer
\usepackage{hyperref}			  % Custom hyperlinks
\usepackage{listings}			  % Code Listings
\usepackage{courier}

\renewcommand{\arraystretch}{1.2} % Extra space in tables
\parindent0mm                     % New paragraphs start without indentation
\setlength{\parskip}{1em}         % And with a blank line in between

% Redefine chapters to remove the "Chapter" word
\titleformat{\chapter}
  {\normalfont\LARGE\bfseries}{\thechapter}{1em}{}
\titlespacing*{\chapter}{0pt}{3.5ex plus 1ex minus .2ex}{2.3ex plus .2ex}

% Setup hyperlink format in document
\hypersetup{
    colorlinks=true, %set true if you want colored links
    linkcolor=blue,  %choose some color if you want links to stand out
    citecolor=blue,  %choose some color if you want lcitation to stand out
    filecolor=black, % etc...
    urlcolor=blue
}

% Define header and footer
\pagestyle{fancy}
\fancyhf{}
\lhead{Kent Milfeld}
\rhead{Maskeraid User Guide}
\lfoot{Maskeraid}
\rfoot{\thepage}

% Define header and footer for first page in chapter
\fancypagestyle{plain}{
\fancyhf{}
\lhead{Kent Milfeld}
\rhead{Maskeraid User Guide}
\lfoot{Maskeraid}
\rfoot{\thepage}
}

% Gray boxes for optional material
\definecolor{LightGrey}{gray}{.85}
\setlength{\fboxrule}{1pt}
\setlength{\fboxsep}{6pt}
\newcommand{\IntroBox}[1]{
  %\fcolorbox[rgb]{0,0,0}{0.95,0.95,0.95}{
    \fcolorbox{black}{LightGrey}{
    \begin{minipage}{0.94\linewidth}
      %\textbf{Introduction}
      #1
    \end{minipage}
  }
}

\lstset{language=C++,
    basicstyle=\ttfamily\scriptsize,
    keywordstyle=\color{blue}\ttfamily,
    stringstyle=\color{red}\ttfamily,
    commentstyle=\color{green}\ttfamily,
    breaklines=true
}

\lstnewenvironment{code}[1][]%
  {\noindent\minipage{\linewidth}\medskip 
   \lstset{basicstyle=\ttfamily\footnotesize,frame=single,#1}}
  {\endminipage}





\begin{document}

\begin{titlepage}
\thispagestyle{empty}	%don't include number on cover
%\begin{flushleft}
%\begin{figure}
%\includegraphics[width=0.2\textwidth]{gplv3-127x51.png}
%\end{figure}
%\end{flushleft}
\verb+ +
\vspace{1em}
\begin{flushright}
\huge\bf Maskeraid v1.2\\
\rule{\textwidth}{4pt}
\large{\bf Mask Aid for Affinity Binding (Scheduling)\\
Document Revision 1.0\\
\today}
\end{flushright}

%Back of cover page (copyright)
\newpage
\thispagestyle{empty}
\begin{flushleft}
Kent Milfeld \\
\verb+milfeld@tacc.utexas.edu+\\
\vspace{0.5em}
High Performance Computing \\
Texas Advanced Computing Center\\
The University of Texas at Austin\\
\vspace{1cm}
Copyright 2016 The University of Texas at Austin.
\end{flushleft}
\newpage
\end{titlepage}

\begin{abstract}

Maskeraid is a set of tools to help researchers discover the binding of their processes(MPI ranks)/threads(OpenMP threads) in a parallel environment. Maskeraid has the following components:

\begin{itemize}
\item Stand-alone executables to report default masks for execution of an OpenMP, MPI or Hybrid in an interactive or batch environment.
\item API for instrumenting applications to report affinity masks from within the program.
\item Utilities: timers, process/thread binder, routine to set core load.
\end{itemize}


Our intention is to create a tool that provides simple-to-understand affinity
information. Bug reports and feedback on usability and improvements are welcome; send to milfeld@tacc.utexas.edu. 


If you use maskeraid, cite:

github.com/TACC/maskeraid, ``Maskeraid: An Aid for Reporting Affinity Masks'', Texas Advanced Computing Center (TACC), Kent F. Milfeld. \cite{maskeraid}

\end{abstract}

\tableofcontents

\chapter{Installation}
Maskeraid is easy to build. A make command will build stand-alone executables, and a library for those who want to instrument their code.

Download the github repository \footnote{\href{https://github.com/TACC/maskeraid}{https://github.com/TACC/maskeraid}} by clicking on the ``Download ZIP'' file, and expand it in a convenient location.

\begin{verbatim}
    unzip  maskeraid-master.zip
\end{verbatim}

You can also clone the git repository:

\begin{verbatim}
    git clone https://github.com/TACC/maskeraid
\end{verbatim}

This will create a top level directory called \verb+maskeraid+, with subdirectories \verb+docs+ and \verb+src+. Change directory to the top level of maskeraid and execute make.

%edit \verb+install.sh+ to reflect your choice for the installation directory. The variables to modify are:

%\begin{table}[h]
%\centering
%\label{tab:env}
%\begin{tabular}{|l|l|l|}
%\hline
%\bf{Variable}	& \bf{Description}                          & \bf{Default}\\\hline
%MASKERAID\_DIR  & Absolute path to installation directory   & . \\\hline
%\end{tabular}
%\end{table}

%Once these variables are set the tool can be installed by using:

%\begin{verbatim}
%    ./install.sh
%\end{verbatim}

The executables will be placed in the \verb+maskeraid/bin+ directory and the library will be placed in \verb+maskeraid/lib+.
%unless the \verb+MASKERAID_DIR=...+ has been modified in the Makefile.

\FloatBarrier
\chapter{Affinity Mask from Stand-Alone Executable}


The standalone executables \verb+omp_whereami+, \verb+mpi_whereami+, and \verb+hybrid_whereami+ are to
be run in an OpenMP, MPI or hybrid environment, respectively, to obtain an output
displaying the affinities for processes/threads in a batch or interactive environment.

\begin{code}[frame=single,breaklines=true,numbers=left,language=C,caption=Executables for evaluating OpenMP MPI and Hybrid masks\label{whereami}]
        export OMP_NUM_THREADS=4; omp_whereami
 
        mpirun -n 8  mpi_whereami
 
        export OMP_NUMPTHREAD=4; mpirun -n 8  hybrid_whereami
\end{code}
 
In batch mode run the whereami executable before your production code, or simply
run the whereami executeable in lieu of the program executable by commenting
it out as shown here:

\begin{code}[frame=single,breaklines=true,numbers=left,language=C,caption=Invoking mask report inside code\label{whereamimpi}]
        #!/bin/bash
        #SLURM  -n 16 -N 1
            ...
        #Batch Script for TACC machine
           ...
        #ibrun ./my_mpi_exec
         ibrun ./mpi_whereami 30
\end{code}

 The whereami puts a load on the ranks for 10 seconds (default), so that \verb+top+ can be run on
 a node to observe the core occupation. You can change the load time (seconds)
 with the \verb+nsec+ argument on the command line.  In the last command above the MPI processes are
 loaded for 30 seconds.  (Execute \verb+top+, and then press the
 1 key without hitting the return key, to change the display to show the percentage
 load on each core (or hardware thread when hyperthreading is on).

%%You can reset the load time by setting setting the MASKERAID\_LOAD\_SECONDS environment
%%variable to a positive integer (units are seconds).

 How to read the output:
 Output 1 is for a Stampede MPI run with 8 MPI processes.

 The rows of the matrix are labeled by the ranks and the columns represent CPU\_ids
 from 0 to ntasks-1, beginning with the first column of the matrix.
 The CPU\_ids are grouped and labeled in units of 10s, and separated by a bar (|).

 A HYPHEN at row, column position (x,y) means that rank x is not to execute on CPU\_id y.
 A NON-HYPHEN character in the matrix means that
 the MPI rank (row value) is allowed to execute in the CPU\_id (column value).
 We use only a single  character to represent occupations, so that we can use
 this tool with many-cores systems with over 50 cores and hundreds of hardware threads.

What do the numbers in the matrix mean?
To make it easy to determine the CPU\_id value in the matrix, the first digit of
CPU\_id (core) number is printed in the matrix.  For CPU\_ids larger than 9, the
"10s" position value can be read from the column header,the groups are separated
by bars (|).  For instance the second set of values {0, 1, 2, 3, 4, 5} along the
diagonal of the matrix represent CPU\_id's { 10, 11, 12, 13, 14, and 15}-- the "10"
in the header "|       10     |" should be added to the unit digits within the matrix to get the
CPU\_id value.


\FloatBarrier
\chapter{Using the Maskeraid Library}

The basic operations used for reporting masking in the \verb+whereami+ programs
were collected into a library, so that users
could instrument there own applications to display the masks of MPI processes
and OpenMP threads.

 

\section{Get Masks Inside a Program}

To report the masking from inside a program, include the maskeraid API routine, \verb+omp_report_mask()+,
\verb+mpi_report_mask()+, or \verb+hybrid_report_mask()+, within an OpenMP parallel region, after MPI has been
initialized, or within an OpenMP region of a hybrid program, respectively.  These functions don't
require any arguments or include files:  

\begin{itemize}
\item omp\_report\_mask()
\item mpi\_report\_mask()
\item hybrid\_report\_mask()
\end{itemize}



However, to report masks for hybrid code, it may be necessary
to initialized MPI with the \verb+MPI_Init_thread()+ routine. View the \verb+whereami+ codes to see how easy
it is to include them in your own program. The snippets below show how they are to be used:



%\begin{lstlisting}[frame=single,breaklines=true,numbers=left,language=C,caption=Invoking mask report inside code\label{code:apimask}]
\begin{code}[frame=single,breaklines=true,numbers=left,language=C,caption=Invoking mask report inside code\label{code:apimask}]
 // Pure OpenMP code
    #pragma omp parallel
    {
       omp_report_mask();
     ...
    }

 // Pure MPI code
    MPI_Init(&argc, &argv);

       mpi_report_mask();
    ...
    MPI_Finalize();

 // Hybrid code
    MPI_Init_thread(&argc, &argv, MPI_THREAD_MULTIPLE, &provided);

    mpi_report_mask();  // helpful: reports ONLY MPI process masks

   #pragma omp parallel private(thrd,nthrds,ierr)
   {
      hybrid_report_mask();
      ...
   }
   ...
   MPI_Finalize();
\end{code}


\section{Useful Utilities}
\begin{itemize}
\item load\_cpu\_nsec( nsec )   --- Puts load on process/thread for nsec seconds
\item map\_to\_cpuid(cpu\_id)   --- Sets process/thread to execute on cpuid
\item gtod\_timer()           --- Easy to use Get Time of Day clock
\item tsc() --- Returns time stamp counter value
\end{itemize}

A thread or process that calls \verb+load_cpu_nsec(int nsec)+ will execute integer
operations (a load) for nsec seconds.  nsec must be zero or a positive integer.  Use the
\verb+cmdln_get_nsec_or_help(int * nsec, int argc, char * argv[])+ function to extract an
integer from the command line for nsec (e.g. a.out 10).


A thread or process that calls \verb+map_to_cpuid(int cpu_id)+ will assign the
calling thread or process to execute on CPU\_id, by setting the appropriate
bit in the scheduling mask.  For example, in a parallel region the unique
thread\_id returned from \verb+omp_get_thread_num()+ can be used in an
arithmetic operation (linear, modulo, etc.) to create a unique CPU\_id to be executed on. 

The function \verb+gtod_timer()+ returns a double precision number with the number
of wall-clock seconds since the previous call.  The first call sets the time
to zero. The function uses the Unix \verb+gettimeofday+ utility, and  can be called from 
C/C++ and Fortran.  See comments in the code for
more details.

The function \verb+tsc()+ returns an 8-byte integer (Unix) time stamp count from
the \verb+rdtsc+ instruction. Use this to capture the difference between the counts for
determining 
the cost a of a small set of operations (instructions, code statements). This is
not meant to be used as as a general timer, since the processor frequency may change.



\section{Other things you should know}
For MPI executables, compiled with IMPI (Intel MPI), 
you can set I\_MPI\_DEBUG=4, and the execution will
report the masking for list of CPU\_ids for each MPI process. 


\FloatBarrier
\addcontentsline{toc}{section}{\bf References}
\begin{thebibliography}{00}


\bibitem{maskeraid} maskeraid is a set of executables and routines for reporting affinity information. \href{https://github.com/tacc/maskeraid}{https://github.com/tacc/maskeraid}


\end{thebibliography}

\end{document}
